\documentclass{article}
\usepackage{polski}
\usepackage[utf8]{inputenc}
\usepackage{natbib}
\usepackage{graphicx}
\usepackage{xcolor}
\usepackage{mathtools}
\usepackage{amssymb}
\usepackage[makeroom]{cancel}
\usepackage{hyperref}
\newcommand{\norm}[1]{\left\lVert#1\right\rVert}

\title{Zadanie 17.24}
\author{Jan Bronicki}
\date{}


\begin{document}

\maketitle
Materiał jest cyklicznie testowany pod kątem wytrzymałości. Wyniki pomiarów podane są w tabeli poniżej, cykle (x) oraz stress (y). Musimy używając regresji wyznaczyć najlepiej pasującą funkcję, dla pomiarów.
\\
Na początku definiujemy:

$$
    x = \log(N), \ \ y = \log(Stress)
$$

\begin{center}
    \begin{tabular}{ c | c c c c c c c}
        x & 0       & 1 & 2       & 3       & 4       & 5       & 6       \\
        \hline
        y & 3.04139 & 3 & 2.96614 & 2.90309 & 2.79588 & 2.74036 & 2.62325
    \end{tabular}
\end{center}

Teraz obliczamy:

$$
    \sum x_{i} =21, \ \ \sum y_{i} = 20.07
$$

$$
    \sum x^{2}_{i} = 91, \ \ \sum x_{i}y_{i} = 58.266
$$

$$
    \bar{x} = 3, \ \ \bar{y} = -0.069427
$$
\\
$$
    a_{0} = 3.0754, \ \ a_{1} = -0.069427
$$


Dzięki temu otrzymujemy taki oto wynik:

$$
    \textcolor{red}{Stress} = 10^{a_{0}} \cdot N^{a_{1}} = \textcolor{red}{10^{3.0754} \cdot N^{-0.069427}}
$$



\end{document}