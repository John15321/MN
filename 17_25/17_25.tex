\documentclass{article}
\usepackage{polski}
\usepackage[utf8]{inputenc}
\usepackage{natbib}
\usepackage{graphicx}
\usepackage{xcolor}
\usepackage{mathtools}
\usepackage{amssymb}
\usepackage[makeroom]{cancel}
\usepackage{hyperref}
\newcommand{\norm}[1]{\left\lVert#1\right\rVert}

\title{Zadanie 17.25}
\author{Jan Bronicki}
\date{}


\begin{document}

\maketitle
Tabela poniżej pokazuje związek pomiędzy temperaturą (x) oraz lepkością oleju (y). Używamy liniowej regresji aby znaleźć najlepszy fit funkcji, dla danych oraz wartości $r^{2}$.
\\
Na początku definiujemy:

$$
    x = \log(Temperature), \ \ y = \log(Viscosity)
$$

\begin{center}
    \begin{tabular}{ c | c c c c}
        x & 1.426   & 1.97    & 2.1729  & 2.4991  \\
        \hline
        y & 0.13033 & -1.0706 & -1.9208 & -3.1249
    \end{tabular}
\end{center}

Teraz obliczamy:

$$
    \sum x_{i} = 8.068, \ \ \sum y_{i} = -5.986
$$

$$
    \sum x^{2}_{i} = 16.881, \ \ \sum x_{i}y_{i} = -13.906
$$

$$
    \bar{x} = 2.017, \ \ \bar{y} = -1.4965
$$
\\
$$
    a_{0} = 4.5815, \ \ a_{1} = -3.0134
$$


Dzięki temu otrzymujemy taki oto wynik:

$$
    \textcolor{red}{Viscosity} = 10^{a_{0}} \cdot Temperature^{a_{1}} = \textcolor{red}{10^{4.5815} \cdot N^{-3.0134}}
$$

$$
    S_{t} = 5.699, \ \ S_{r} = 0.13752, \ \ \textcolor{red}{r^{2} = 0.97570}
$$


Wynik lepkości:

$$
    Viscosity = 10^{4.5815} \cdot N^{-3.0134}
$$

$$
    r^{2} = 0.9757
$$

\end{document}