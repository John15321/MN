\documentclass{article}
\usepackage{polski}
\usepackage[utf8]{inputenc}
\usepackage{natbib}
\usepackage{graphicx}
\usepackage{xcolor}
\usepackage{mathtools}
\usepackage{amssymb}
\usepackage[makeroom]{cancel}
\usepackage{hyperref}
\newcommand{\norm}[1]{\left\lVert#1\right\rVert}

\title{Zadanie 16.12}
\author{Jan Bronicki}
\date{}


\begin{document}

\maketitle

Należy zaprojektować trójkątny kanał przepuszczający odpady chemiczne.

Dla stałej wartości powierzchni przekroju kanału $A$, można napisać funkcję:
$$
    A=\frac{wd}{2} \rightarrow d=\frac{2A}{w}
$$

Parametr $p$ może zostać zapisany jako:

$$
    p=w+2\sqrt{\left(\frac{w}{2}\right)^{2}+d^{2}}
$$

Może to zostać wyrażone jako funkcja $w$. Wtedy taka jest funkcja która należy z minimalizować:

$$
    f(w) = w+2\sqrt{\left(\frac{w}{2}\right)^{2}+\left(\frac{2A}{w}\right)^{2}}
$$

Zrżnicowana funkcja $f(w)$:

$$
    f^{\prime}(w)=1+\frac{w^{4}-16A^{2}}{w^{2}\sqrt{w^{4}+16A^{2}}}
$$


$f^{\prime}$ osiąga minimum kiedy wynosi $0$, tak więc:


$$
    f^{\prime}(w)=0
$$
$$
    \frac{w^{4}-16A^{2}}{w^{2}\sqrt{w^{4}+16A^{2}}}=-1
$$

Z tego otrzymujemy:

$$
    w^{2}=\frac{4}{\sqrt{3}}A
$$

Oraz

$$
    \tan \theta = \frac{d}{\frac{w}{2}}=\frac{\frac{2A}{w}}{\frac{w}{2}}=\frac{4A}{w^{2}}
$$

\newpage

Ponieważ $\theta$ ma być takie aby parametr był jaknajmniejszy:

$$
    \theta = \arctan \left(\frac{4A}{\frac{4A}{\sqrt{3}}}\right)=\arctan \sqrt{3}=60^{\circ}
$$

Tak więc szerokość i wysokość potrzebnego trójkąto to:

$$
    w=\frac{2\sqrt{A}}{\sqrt[3]{4}}, \ d=\frac{\sqrt{3A}}{\sqrt[3]{4}}, \ \theta=60^{\circ}
$$

\end{document}